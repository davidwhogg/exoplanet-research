\documentclass[12pt, preprint]{aastex}

\newcounter{affil}
\newcommand{\project}[1]{\textsl{#1}}
\newcommand{\Kepler}{\project{Kepler}}

\begin{document}

\title{%
  Single transits in the NASA \Kepler\ data\\
  and the abundance of extra-solar Jupiter analogs}

\author{%
  So Hattori\altaffilmark{\ref{nyuad}},
  Dan Foreman-Mackey\altaffilmark{\ref{ccpp}},
  David W. Hogg\altaffilmark{\ref{ccpp}, \ref{cds}, \ref{mpia}},
  and others}
\refstepcounter{affil}\label{nyuad}\altaffiltext{\theaffil}{%
  Department of Physics, New York University Abu Dhabi}
\refstepcounter{affil}\label{ccpp}\altaffiltext{\theaffil}{%
  Center for Cosmology and Particle Physics,
  Department of Physics, New York University}
\refstepcounter{affil}\label{cds}\altaffiltext{\theaffil}{%
  Center for Data Science, New York University}
\refstepcounter{affil}\label{mpia}\altaffiltext{\theaffil}{%
  Max-Planck-Institut f\"ur Astronomie}

\begin{abstract}
In the Solar System, gravitational perturbations from Jupiter are
critical in setting the distribution of planets (in mass and
semi-major axis).
If we want to understand Solar-System formation, we need to understand
planets like Jupiter.
The \Kepler\ mission operated for 4.1 years; long-period planets
analogous to Jupiter will not transit multiple times in the data.
Searches for periodic signals in the \Kepler\ data will not reveal
Jupiter analogs.
Here we search the light-curves of dwarf stars in [temperature ranges]
in the \Kepler\ data for single transits that have roughly the right
duration and depth to be single transits of long-period gas-giant
planets.
We use probabilistic hypothesis tests against alternatives to rule out
instrument-generated artifact false-positives, and probabilistic
inference to obtain likelihood information about planet radius and
orbital period.
Each transit we find gives exceedingly low signal-to-noise information
about period but we combine the likelihood information using a
hierarchical model to infer properties of the full distribution of
long-period, large planets.
We find ZZZ.
We discuss the problem of astrophysical false positives, but the
conservative position is to treat this measurement as an upper limit.
\end{abstract}

hello world.

\end{document}
