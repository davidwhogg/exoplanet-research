\documentclass[12pt, preprint]{aastex}
\input{vc}

\newcounter{affil}
\newcommand{\project}[1]{\textsl{#1}}
\newcommand{\Kepler}{\project{Kepler}}

\begin{document}

\title{%
  Single transits in the NASA \Kepler\ data\\
  and the abundance of extra-solar Jupiter analogs}

\author{%
  So Hattori\altaffilmark{\ref{nyuad}},
  Daniel Foreman-Mackey\altaffilmark{\ref{ccpp}},
  David W. Hogg\altaffilmark{\ref{ccpp}, \ref{cds}, \ref{mpia}},
  and others}
\refstepcounter{affil}\label{nyuad}\altaffiltext{\ref{nyuad}}{%
  Department of Physics, New York University Abu Dhabi}
\refstepcounter{affil}\label{ccpp}\altaffiltext{\ref{ccpp}}{%
  Center for Cosmology and Particle Physics,
  Department of Physics, New York University}
\refstepcounter{affil}\label{cds}\altaffiltext{\ref{cds}}{%
  Center for Data Science, New York University}
\refstepcounter{affil}\label{mpia}\altaffiltext{\ref{mpia}}{%
  Max-Planck-Institut f\"ur Astronomie}

\date{DRAFT --- not ready for distribution --- \texttt{git hash \githash\ (\gitdate)}}

\begin{abstract}
In the Solar System, gravitational perturbations from Jupiter are
critical in setting the distribution of planets (in mass and
semi-major axis).
If we want to understand Solar-System formation, we need to understand
planets like Jupiter.
The \Kepler\ mission operated for only 4.1 years; searches for
periodic signals in the \Kepler\ data will not reveal Jupiter analogs.
Here we search the light-curves of dwarf stars in [temperature ranges]
in the \Kepler\ data for single transits that have roughly the right
duration and depth to be single transits of long-period gas-giant
planets.
We use hypothesis tests against alternatives to rule out
instrument-generated artifact false-positives, and probabilistic
inference to obtain likelihood information about planet radius and
orbital period.
Each transit we find gives exceedingly low signal-to-noise information
about period but we combine the likelihood information using a
hierarchical model to infer properties of the full distribution of
long-period, large planets.
We find [some results].
We discuss the problem of astrophysical false positives, but the
conservative position is to treat this measurement as an upper limit.
\end{abstract}

\section{Introduction}

...The point that Jupiter is in charge in the Solar System

...There are long-period planets known from RV data with very
uncertain periods and masses.  Cite Knutson here.

...There are single transits known in the \Kepler\ data but no
published list.  Is that true?

...Why is it worth searching?  Calculation for a world in which every
star has a perfect Jupiter analog.

\section{Search and Characterization}

Our search for single transits proceeds in three stages.
First we look for places where the data are better explained by a
top-hat or square-wave or box downward deviation from a constant level
is a better explanation of the data.
This ``box search'' produces ``box candidates''.
Second we perform a standard hypothesis test between a detailed
transit model (a full model including limb darkening) and some
``artifact'' models that attempt to model typical spacecraft-induced
features seen commonly in the \Kepler\ data.
The box candidates that pass this hypothesis test are called ``transit
candidates''.
Third we perform MCMC sampling with fairly uninformative interim
priors to obtain posterior (and, indirectly, likelihood) information
about the transit parameters that explain each of the transit
candidates.
This characterization step produces a $K$-element sampling in planet
radius, period, and impact parameter for each transit candidate.

...Box search

...Hypothesis tests

...Characterization

\section{Population Model}

...This section only needs to be written if we have transit
candidates!

The information coming from any individual transit candidate, about
period, say, is extremely small.
That is, we can't just use the ``central values'' of the likelihood
functions or posterior pdfs and expect to get good inferences about
the population of Jupiter-like planets.

The only principled way to use noisy information at the
individual-object level and nonetheless make inferences...

\section{Discussion}

...Recap

...Assumptions:  List each one and discuss.

...Astrophysical false positives?

...Circular orbits?

\end{document}
