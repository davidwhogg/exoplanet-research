\documentclass[a4paper]{article}

\usepackage[english]{babel}
\usepackage[utf8]{inputenc}
\usepackage{amsmath}
\usepackage{graphicx}
\usepackage[colorinlistoftodos]{todonotes}

\title{Jupiter like exoplanets}

\author{Soichiro Hattori}

\date{\today}

\begin{document}
\maketitle

\section{Jupiter-size transit depth and width}
\subsubsection*{Transit depth}
Using simple geometry(I will need to write this up with my notes later) one can roughly approximate the depth of a transit $\delta$ for a planet of radius $R_p$ orbiting a star of radius $R_s$ to be $$\delta=(\frac{R_p}{R_s})^2$$
Given that a Jupiter-size transit is what is needed, it is possible to roughly approximate the transit depth by the following, $$\delta_j=(\frac{R_j}{R_s})^2$$ where $R_j$ is the radius of Jupiter ($6.9173\times10^7$m), and $R_s$ is the radius of the sun ($6.955\times10^8$m). From the calculation, $$\delta_j=0.00989188$$
this is a unitless number that represents the percentage of how much the lightcurve will dip during the transit. For a Jupiter-size planet, it is roughly $1\%$

\subsubsection*{Transit Width}
(Derivation will be written later.)
The transit width $T$, with units in days, is given approximately by the equation $$T=\frac{R_sP}{\pi r}$$ where $R_s$ is the radius of the star, $P$ is the period of the orbit of the planet in days, and $r$ is the distance from the star to the planet. To approximate the values for the Sun and Jupiter will be used. The time it takes for Jupiter to complete a full orbit around the sun is $P=4332.8201$ days, and the distance between the Sun and Jupiter is $r=7.9062\times10^{11}$m. Plugging these numbers into the equation,  $$T=1.21325$$ where the unit is in days. Converting to hours, it is $29.118$ hours.
\end{document}